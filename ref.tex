\documentclass{exam}

\usepackage{amsmath} 
\usepackage{txfonts}

\title{Reeks-genererende polynomen}
\author{Timothy van der Valk}

\begin{document}

\maketitle 

% Vraag 1.
\begin{questions}
\begin{parts}

\question Laat $f(x) = c_2 x^2 + c_1 x + c_0$ met $c_0,c_1,c_2 \in \mathbb{R}$. 
Laat $s_i$ een reeks in $\mathbb{R}$ zijn, bijvoorbeeld $(1, 2, 3, 4)$. Een polynoom $f$ \textit{genereert} de reeks
$s_i$ als $f(i) = s_i$ voor $0 \leq i \leq n$ met een zekere $n \geq 0$.
De polynoom $f(x) = x^2$ genereert $(0, 1, 4)$ omdat $f(0) = 0, f(1) = 1, f(2) = 4$.

\part Geef een polynoom die $(1, 2, 3)$ genereert.

\part Geef een polynoom die $(1, 1, 2)$ genereert.

\part Geef een polynoom die een willekeurige reeks $(s_1, s_2, s_3)$ genereert.

\part Geef de matrix $A_3$ zodat $A_3 \cdot 
\begin{bmatrix} c_0 \\ c_1 \\ c_2 \end{bmatrix} = 
\begin{bmatrix} f(0) \\ f(1) \\ f(2) \end{bmatrix}.$

\part Bereken de inverse van $A_3$. {\it(Hint: zie 1.3)}

\part Bewijs dat de coefficienten $c_i$ uniek zijn in het genereren van de reeks $s_n$.


\question We hebben nu polynomen gevonden die reeksen van drie getallen genereren. 
Dit process kan worden uitgebreid voor reeksen met $n$ getallen.
Laat $f_n$ een polynoom met graad $n - 1$ gegeven worden door
\begin{equation}
	f_n(x) = \sum_{i=0}^n{c_i x^i} \quad.
\end{equation}

Met $c_i \in \mathbb{R}$. Laat verder $\underline{x}$ de vector zijn met de coefficienten van $f$ zodat $x_i = c_i$. 
Laat $\underline{b}$ de vector zijn met de functiewaarden van $f$ met $b_i = f(i)$ voor $0 \leq i \leq n$.

\part Geef een formule voor de matrix $A_n$ waarvoor geldt 
$A_n \cdot \underline{x} = \underline{b}$. 

De matrices $A_n$ zijn inverteerbaar voor alle $n \geq 1$.

\part {Geef een tegenvoorbeeld die laat zien dat reeks-genererende polynomen met verschillende graad dezelfde reeks kunnen maken tot op zekere hoogte.}

\part Bewijs dat twee reeks-genererende polynomen van dezelfde graad $n$ uniek zijn.\geq 1$.

\part \textbf{[*Lastig]} Bewijs dat $A_n$ inverteerbaar is.

\end{parts}
\end{questions}

\end{document}
