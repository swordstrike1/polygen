\documentclass{exam}

\usepackage{amsmath} 
\usepackage{txfonts}
\renewcommand\partlabel{\thepartno.}% Default is (\thepartno)
\title{Reeks-genererende polynomen}
\author{Timothy van der Valk}

\begin{document}
\maketitle

% Vraag 1.
\begin{questions}

\question Laat $f(x) = c_2 x^2 + c_1 x + c_0$ met $c_0,c_1,c_2 \in \mathbb{R}$. 
Laat $s_i \in \mathbb{Z}$ een eindige reeks zijn met $n$ termen. Een polynoom $f$ \textit{genereert}
deze reeks als $f(i) = s_i$. Dus $f$ geeft een bijectie van een 
deelverzameling van $\mathbb{Z}$ naar de reeks $s_i$.

Voorbeeld: De polynoom $f(x) = x^2$ genereert $(0, 1, 4)$ omdat $f(0) = 0, f(1) = 1, f(2) = 4$.

\begin{parts}
\part Geef polynomen die $(1, 2, 3)$ en $(1, 1, 2)$ genereren.

\part Geef een polynoom die de reeks $(s_1, s_2, s_3)$ genereert.

\part Geef de matrix $A_3$ zodat $A_3 \cdot 
\begin{bmatrix} c_0 \\ c_1 \\ c_2 \end{bmatrix} = 
\begin{bmatrix} f(0) \\ f(1) \\ f(2) \end{bmatrix}.$

\part Bereken de inverse van $A_3$. {\it(Hint: zie 1b)}


\end{parts}

\question Deze vraag gaat over de verzameling van alle reeks-genererende
polynomen.

\begin{parts}
\part Bewijs dat de reeks-genererende polynoom $f$ voor de reeks $(s_1, s_2, s_3)$
uniek is.

\part Geef de verzameling van reeksen $(s_1, s_2, s_3)$ waarvan de genererende
polynoom $f$ coefficienten heeft in $\mathbb{Z}$.

\part Bereken het aantal reeksen $(s_1, s_2, s_3)$ met $0 \leq s_i < 10$ waarvan
de generende polynomen $f$ coefficienten hebben in $\mathbb{Z}$.

\end{parts}


\question We kunnen reeks-genererende polynomen uitbreiden voor reeksen met
willkeurige lengte door hogere orde polynomen te gebruiken.

\begin{parts}

\part Geef een formule voor de term $x_{i,j}$ van de matrix $A_n$ waarvoor geldt
$A_n \cdot
\begin{bmatrix} c_0 \\ \vdots \\ c_{n-1} \end{bmatrix} = 
\begin{bmatrix} f(0) \\ \vdots \\ f(n-1) \end{bmatrix}$. {\it(Hint: zie 1c)}

\part Geef een voorbeeld van twee polynomen met verschillende graad die 
tot zekere hoogte dezelfde reeks genereren. Dus reeks-genererende polynomen zijn
uniek op graad na.

\part \textbf{[*Taai]} Bewijs dat $A_n$ inverteerbaar is.

\end{parts}

\end{questions}

\end{document}
